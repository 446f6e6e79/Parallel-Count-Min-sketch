% Pacchetti essenziali
\usepackage{amsmath}
\usepackage{graphicx}
\usepackage{float}
\usepackage{comment}
\usepackage{listings}
\usepackage{xcolor}
\usepackage[hidelinks]{hyperref}
\usepackage{enumitem}
\usepackage{titlesec}
\usepackage{algorithm}
\usepackage{algpseudocode}
\usepackage{tcolorbox} 
\usepackage{fancyhdr} % Aggiunto per intestazioni personalizzate

% Definition of the command for the month
\newcommand{\MonthName}{%
  \ifcase\month\or January\or February\or March\or April\or May\or June\or July\or August\or September\or October\or November\or December\fi}

  % Configurazione di codeblocks con il pacchetto listings
\lstset{
  basicstyle=\ttfamily\small,
  keywordstyle=\color{blue},
  commentstyle=\color{green!50!black},
  stringstyle=\color{red},
  showstringspaces=false,
  breaklines=true,
  frame=single, % Aggiunge un bordo attorno al codice
  numbers=left, % Numeri di riga a sinistra
  numberstyle=\tiny\color{gray}, % Stile dei numeri di riga
  stepnumber=1, % Numero di righe da saltare tra i numeri
  numbersep=-5pt, % Distanza tra i numeri di riga e il codice
  tabsize=2, % Dimensione del tab
  captionpos=b, % Posizione della didascalia
}

% Configurazione dell'intestazione
\pagestyle{fancy}
\fancyhf{} % Pulisce le intestazioni e i piè di pagina
\fancyhead[L]{\leftmark} % Sezione corrente a sinistra
\fancyhead[R]{\thepage} % Numero di pagina a destra

% Comandi personalizzati per definizioni
\newcommand{\definition}[2]{%
  \paragraph{#1:} #2%
}

% Alternativa con enfasi sul termine
\newcommand{\defterm}[2]{%
  \paragraph{\textbf{#1}:} #2%
}

% Ambiente per liste di definizioni
\newenvironment{definitions}{%
  \begin{description}[font=\normalfont\bfseries\large, leftmargin=0pt, labelindent=0pt]
}{%
  \end{description}
}

% Definizione con box colorato (opzionale)
\newtcolorbox{defbox}[1]{
  colback=blue!5!white,
  colframe=blue!75!black,
  title=#1,
  fonttitle=\bfseries
}